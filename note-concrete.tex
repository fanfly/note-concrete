\documentclass[11pt]{article}
\usepackage[a4paper,top=2cm,bottom=3cm,left=1.5cm,right=1.5cm]{geometry}
\usepackage{titling}
\usepackage{amsmath}
\usepackage{amssymb}
\usepackage{amsthm}
\usepackage{tikz}
\usepackage{thmtools}
\usepackage[shortlabels]{enumitem}
\usepackage{abstract}
\usepackage{hyperref}

\title{Mathematical Analysis of Algorithms}
\date{Spring 2021}

% bold math
\makeatletter
\g@addto@macro\bfseries{\boldmath}
\makeatother

% remove abstract title
\renewcommand{\abstractname}{}
\renewcommand{\absnamepos}{empty}

% style of links
\hypersetup{colorlinks,linkcolor=black}

% set theorem style
\declaretheoremstyle[
  spaceabove=6pt, spacebelow=6pt,
  headfont=\normalfont\bfseries,
  notefont=\normalfont\bfseries,
  bodyfont=\normalfont\upshape,
  postheadspace=0.5em
]{custom}

% set qed symbol
\renewcommand{\qedsymbol}{$\blacksquare$}

% types of theorems
\declaretheorem[style=custom,parent=section]{definition}
\declaretheorem[style=custom,sibling=definition]{example}
\declaretheorem[style=custom,sibling=definition]{theorem}
\declaretheorem[style=custom,sibling=definition]{fact}
\declaretheorem[style=custom,sibling=definition]{algorithm}

% use bold fonts to emphasize
\DeclareTextFontCommand{\emph}{\bfseries}

\newcommand{\NN}{\mathbb{N}}
\newcommand{\ZZ}{\mathbb{Z}}
\newcommand{\QQ}{\mathbb{Q}}
\newcommand{\RR}{\mathbb{R}}

\begin{document}

% title
\begin{center}
  \LARGE \bfseries \thetitle, \thedate
\end{center}

\begin{abstract}
  This note is taken for the course "Mathematical Analysis of Algorithms", which is instructed by Wen-Chin Chen in Spring 2021.
\end{abstract}

\tableofcontents

\section{February 26, 2021}
\subsection{Course Introduction}
This course focuses on \emph{concrete mathematics}, which is a blend of continuous and discrete mathematics.

\begin{itemize}
  \item Scoring: Homeworks ($100\%$).
  \item Textbook: \textsl{Concrete Mathematics: A Foundation for Computer Science}, by Ronald Graham, Donald Knuth and Oren Patashnik (1989).
\end{itemize}

\subsection{Tower of Hanoi}
The \emph{Tower of Hanoi} consists of three pegs, and we are given $n$ disks, which are initially stacked in decreasing size on one of the three pegs.
The goal is to move the entire tower to one of the other pegs.
In each step, only one disk can be moved, and a larger one cannot be moved onto a smaller one.
We need to determine the minimum number $T_n$ of moves needed to reach the goal.

\begin{theorem}
  $T_n = 2^n - 1$ for all $n \in \NN$.
\end{theorem}
\begin{proof}
  Clearly $T_0 = 0$.
  To move $n$ disks for $n \geq 1$, one ca
  n transfer the $n - 1$ smallest disks to a single peg, then move the largest one to a different peg, and then transfer the $n - 1$ smallest ones back onto the largest one.
  Thus $T_n \leq 2T_{n-1} + 1$ for $n \geq 1$.
  \par Note that we have to move the largest disk at some point, and before that the $n-1$ smallest disks should be transferred to a single peg, requiring $T_{n-1}$ moves.
  Then we move the largest disk.
  While one may move the largest disk more than once, after moving the largest disk for the last time we must transfer the $n-1$ smallest disks back onto the largest one, requiring $T_{n-1}$ moves.
  Thus $T_n \geq 2T_{n-1} + 1$ for $n \geq 1$.
  \par It follows that for $n \geq 1$, we have $T_n = 2T_{n-1} + 1$, implying $T_n + 1 = 2(T_{n-1} + 1)$.
  Thus, $T_n = 2^n(T_0 + 1) - 1 = 2^n - 1$ for all $n \in \NN$.
\end{proof}

\subsection{Josephus Problem}
In the \emph{Josephus problem}, there are $n$ people numbered from $1$ to $n$ standing in a circle.
Every second remaining person is eliminated until only one survives.
We need to determine the number $J_n$ of the survivor.
For instance, we have $J_{10} = 5$ since
\begin{align*}
  (1, 2, 3, 4, 5, 6, 7, 8, 9, 10)
  &\to (3, 4, 5, 6, 7, 8, 9, 10, 1) \\
  &\to (5, 6, 7, 8, 9, 10, 1, 3) \\
  &\to (7, 8, 9, 10, 1, 3, 5) \\
  &\to (9, 10, 1, 3, 5, 7) \\
  &\to (1, 3, 5, 7, 9) \\
  &\to (5, 7, 9, 1) \\
  &\to (9, 1, 5) \\
  &\to (5, 9) \\
  &\to (5).
\end{align*}

\begin{theorem}
  Let $m$ and $\ell$ be integers with $m \geq 0$ and $0 \leq \ell < 2^m$.
  Then $J_{2^m + \ell} = 2\ell + 1$.
\end{theorem}
\begin{proof}
  Clearly $J_1 = 1$.
  We have $J_{2n} = 2J_n - 1$ for $n \geq 1$ since
  \begin{equation*}
    (1, 2, 3, 4, \dots, 2n-1, 2n) \to^n (1, 3, \dots, 2n-1).
  \end{equation*}
  Also we have $J_{2n+1} = 2J_n + 1$ for $n \geq 1$ since
  \begin{equation*}
    (1, 2, 3, 4, \dots, 2n, 2n+1) \to^{n+1} (3, 5, \dots, 2n+1).
  \end{equation*}
  \par Now we show that $J_{2^m + \ell} = 2^\ell + 1$ for $m \geq 0$ and $0 \leq \ell < 2^m$ by induction on $m$.
  For the induction basis, let $m = 0$.
  In this case we must have $\ell = 0$, and $J_{2^0} = 1 = 2 \cdot 0 + 1$ holds.
  Now suppose that $J_{2^{m-1} + \ell} = 2^\ell + 1$ holds for any $0 \leq \ell < 2^{m-1}$.
  If $\ell = 2k$ for some integer $k$, we have
  \begin{equation*}
    J_{2^m + \ell}
    = 2J_{2^{m-1} + k} - 1
    = 2(2k + 1) - 1
    = 2(2k) + 1
    = 2\ell + 1.
  \end{equation*}
  If $\ell = 2k + 1$ for some integer $k$, we have
  \begin{equation*}
    J_{2^m + \ell}
    = 2J_{2^{m-1} + k} + 1
    = 2(2k + 1) + 1
    = 2\ell + 1.
  \end{equation*}
  This completes the proof.
\end{proof}

\end{document}
